%%%%%%%%%%%%%%%%%%%%%%%%%%%%%%%%%%%%%%%%%%%%%%%%%%%%
% Artifact Appendix Template for EuroSys'22 AE
%
% this document has a maximum length of 2 pages.
%%%%%%%%%%%%%%%%%%%%%%%%%%%%%%%%%%%%%%%%%%%%%%%%%%%%

\appendix
\section{Artifact Appendix}
% \textit{This artifact appendix is meant to be a self-contained document which
% describes a roadmap for the evaluation of your artifact. It should include a
% clear description of the hardware, software, and configuration requirements. In
% case your artifact aims to receive the functional or results reproduced badge,
% it should also include the major claims made by your paper and instructions on
% how to reproduce each claim through your artifact. Linking the claims of your
% paper to the artifact is a necessary step that ultimately allows artifact
% evaluators to reproduce your results.}

% \textit{Please fill all the mandatory sections, keeping their titles and
% organization but removing the current illustrative content, and remove the
% optional sections where those do not apply to your artifact.}

%%%%%%%%%%%%%%%%%%%%%%%%%%%%%%%%%%%%%%%%%%%%%%%%%%%%%%%%%%%%%%%%%%%%%
\subsection{Abstract}
% {\em [Mandatory]}
% {\em Provide a short description of your artifact.}

As introduced in the paper, we measure and characterize the GFW's new
system for censoring fully encrypted traffic. We find that, 
instead of directly defining what fully encrypted traffic is, the
censor applies crude but efficient heuristics to exempt traffic
that is unlikely to be fully encrypted traffic; it then blocks the
remaining non-exempted traffic. These heuristics are based
on the fingerprints of common protocols, the fraction of set
bits, and the number, fraction, and position of printable ASCII
characters. 
In this artifact,
we provide the data and code to reproduce our results.

%%%%%%%%%%%%%%%%%%%%%%%%%%%%%%%%%%%%%%%%%%%%%%%%%%%%%%%%%%%%%%%%%%%%%
\subsection{Description \& Requirements}

% \textit{[Mandatory] This section should list all the information necessary to
% recreate the same experimental setup you have used to run your artifact. Where
% it applies, the minimal hardware and software requirements to run your artifact.
% It is also very good practice to list and describe in this section benchmarks
% where those are part of, or simply have been used to produce results with, your
% artifact.}

\subsubsection{Security, privacy, and ethical concerns}
% \textit{[Mandatory] Describe any risk for evaluators while executing your
% artifact to their machines security, data privacy or others ethical concerns.
% This is particularly important if destructive steps are taken or security
% mechanisms are disabled during the execution.}
As detailed in the ethics section of our paper,
our measurement tools have already employed the best practices by default.
\subsubsection{How to access}
% {\em [Mandatory]} \textit{Describe here how to access your artifact. If you are
% applying for the Artifacts Available badge, the archived copy of the artifacts
% must be accessible via a stable reference or DOI. For this purpose, we recommend
% Zenodo, but other valid hosting options include institutional and third-party
% digital repositoriesValid hosting options include institutional repositories and
% third-party digital repositories (e.g., Zenodo, FigShare, Dryad, Software
% Heritage, GitHub, or GitLab — not personal webpages). For repositories that can
% evolve over time (e.g., GitHub), a stable reference to the evaluated version
% (e.g., a URL pointing to a commit hash or tag) rather than the evolving version
% reference (e.g., a URL pointing to a mere repository) is required. Note that the
% stable reference provided at submission time is for the purpose of Artifact
% Evaluation. Since the artifact can potentially evolve during the evaluation to
% address feedback from the reviewers, another (potentially different) stable
% reference will be later collected for the final version of the artifact (to be
% included here for the camera-ready version)}

The artifact is available on GitHub:
\url{https://github.com/69mu/usenixsecurity23-artifact}.


\subsubsection{Hardware dependencies}
%{\em [Mandatory]} \textit{Describe any specific hardware features required to
%evaluate your artifact (vendor, CPU/GPU/FPGA, number of processors/cores,
%microarchitecture, interconnect, memory, hardware counters, etc). If your
%artifact requires special hardware, please provide instructions on how to gain
%access to the hardware. For example, provide private SSH keys to access the
%machines remotely. Please keep in mind that the anonymity of the reviewers needs
%to be maintained and you may not collect or request personally identifying
%information (e.g., eMail, name, address). [Simply write "None." where this does
%not apply to your artifact.]}

We have prepared a VPS in China and a VPS in the US,
on which the AE reviewers can perform remote experiments.
%
The VPS in China is located in AlibabaCloud Beijing Datacenter (AS37963),
which uses one core of Intel Xeon Platinum 8163 and 1GB RAM.
The VPS in the US is located in DigitalOcean San Francisco Datacenter (AS14061),
which uses one core of Intel DO-Regular and 1GB RAM.
To SSH into the VPSes, reviewers need to install the provided credentials,
as detailed in \texttt{artifacts/setup-vps/README.md}, 

\subsubsection{Software dependencies}
%{\em [Mandatory]} \textit{Describe any specific OS and software packages
%required to evaluate your artifact. This is particularly important if you share
%your source code and it must be compiled or if you rely on some proprietary
%software that you cannot include in your package. In such a case, you must
%describe how to obtain and to install all third-party software, data sets, and
%models. [Simply write "None." where this does not apply to your artifact.]}

The VPS in China runs Ubuntu 22.04.2 LTS (GNU/Linux 5.15.0-56-generic x86\_64).
The VPS in the US runs Ubuntu 20.04.3 LTS (GNU/Linux 5.4.0-88-generic x86\_64).
The following tools and environment are required:

\begin{itemize}
    \item GNU make utility
    \item Go 1.17+
    \item Python 3.8+
\end{itemize}

In particular, the two VPSes do not require Go environment. 
As detailed in \texttt{README},
reviewers may compile the Go code on their machines and copy the binaries to the VPSes.

\subsubsection{Benchmarks}
% {\em [Mandatory]} \textit{Describe here any data (e.g., data-sets, models,
% workloads, etc.) required by the experiments with this artifact reported in your
% paper.} \textit{[Simply write "None." where this does not apply to your
% artifact.]}
None.

%%%%%%%%%%%%%%%%%%%%%%%%%%%%%%%%%%%%%%%%%%%%%%%%%%%%%%%%%%%%%%%%%%%%%
\subsection{Set-up}

%{\em [Mandatory]} \textit{This section should include all the installation and
%configuration steps required to prepare the environment to be used for the
%evaluation of your artifact.}

As detailed in \texttt{artifacts/setup-vps/README.md}, 
we have prepared a VPS in China and a VPS in the US,
on which the AE reviewers can perform remote experiments.
Since we have installed the dependencies and required tools on the VPSes,
all reviewers need to do is using the provided credentials to SSH into the VPSes
to run experiments.
We also provide one-click scripts, allowing reviewers to initialize test-ready VPSes themselves.
Be very cautious that running setup scripts will disrupt other reviewers' ongoing experiments.

\subsubsection{Installation}
% {\em [Mandatory]} \textit{Instructions to download and install dependencies as
% well as the main artifact. After these steps the evaluator should be able to run
% a simple functionality test.}

\begin{itemize}
    \item Set up the Go environment: \url{https://go.dev/dl/}.
    \item Retrieve the artifact:
    \texttt{git clone \url{https://github.com/69mu/usenixsecurity23-artifact.git}}.
    \item Compile the experiment tools and install them to the VPS in China:
    \texttt{cd artifacts/util \&\& make install}.
\end{itemize}

\subsubsection{Basic Test}
%{\em [Mandatory]} \textit{Instructions to run a simple functionality test. Does
%not need to run the entire system, but should check that all required software
%components are used and functioning fine. Please include the expected successful
%output and any required input parameters.}

First login to the VPS in China using the provided credentials:
\texttt{ssh usenix-ae-client-china}.

Then send random probes to the port $80$ of the $\$serverIP$ with the following command:
\texttt{echo \$serverIP | ./utils/affected-norand -p 80 -log /dev/null}.

The program outputs in CSV format. 
If the \texttt{affected} field is \texttt{True},
the blocking is successfully triggered.


%%%%%%%%%%%%%%%%%%%%%%%%%%%%%%%%%%%%%%%%%%%%%%%%%%%%%%%%%%%%%%%%%%%%%
\subsection{Evaluation workflow}
% {\em [Mandatory for Artifacts Functional \& Results Reproduced, optional for
% Artifact Available]} \textit{This section should include all the operational
% steps and experiments which must be performed to evaluate if your your artifact is
% functional and to validate your paper's key results and claims. For that
% purpose, we ask you to use the two following subsections and cross-reference the
% items therein as explained next.}

\subsubsection{Major Claims}
% {\em [Mandatory for Artifacts Functional \& Results Reproduced, optional for
% Artifact Available]} \textit{Enumerate here the major claims (Cx) made in your
% paper. Follows an example:}\\

\begin{compactdesc}

    \item[(C1):] \textit{}
    The GFW blocks random traffic.
    This is supported by the experiment (E1).

    \item[(C2):] \textit{}
    The GFW exempts a connection if
    the first TCP packet $\mathtt{pkt}$ satisfies:
    $\frac{\mathit{popcount}(\mathtt{pkt})}{\mathit{len}(\mathtt{pkt})} \le 3.4$ or 
    $\frac{\mathit{popcount}(\mathtt{pkt})}{\mathit{len}(\mathtt{pkt})} \ge 4.6$. 
    This is supported by the experiment (E2) described in Section 4.1 of the paper. 
    This detection rule is introduced in Algorithm 1 (Ex1)
    and illustrated in Figure 1.d.

    \item[(C3):] \textit{}
    The GFW exempts a connection if
    the first six (or more) bytes of the first TCP data packet $\mathtt{pkt}$ are 
    $[\mathtt{0x20},\mathtt{0x7e}]$. 
    This is supported by the experiment (E3) described in Section 4.2 of the paper. 
    This detection rule is introduced in Algorithm 1 (Ex2)
    and illustrated in Figure 1.a.

    \item[(C4):] \textit{}
    The GFW exempts a connection if 
    the first TCP data packet $\mathtt{pkt}$ has 
    more than 50\% of $\mathtt{pkt}$'s bytes in $[\mathtt{0x20},\mathtt{0x7e}]$. 
    This is supported by the experiment (E4) described in Section 4.2 of the paper. 
    This detection rule is introduced in Algorithm 1 (Ex3)
    and illustrated in Figure 1.b.

    \item[(C5):] \textit{}
    The GFW exempts a connection if 
    the first TCP data packet $\mathtt{pkt}$ has 
    more than 20 contiguous bytes in
    $[\mathtt{0x20},\mathtt{0x7e}]$. 
    This is supported by the experiment (E5) described in Section 4.2 of the paper. 
    This detection rule is introduced in Algorithm 1 (Ex4)
    and illustrated in Figure 1.c.

    \item[(C6):] \textit{}
    The GFW exempts a connection if 
    the first few bytes of the first TCP data packet $\mathtt{pkt}$
    match the protocol fingerprint for TLS or HTTP.
    This is supported by the experiment (E6) described in Section 4.3 of the paper. 
    This detection rule is introduced in Algorithm 1 (Ex5)
    and illustrated in Figure 1.e.

\end{compactdesc}

\subsubsection{Experiments}
% {\em [Mandatory for Artifacts Functional \& Results Reproduced, optional for
% Artifact Available]} \textit{Link explicitly the description of your experiments
% to the items you have provided in the previous subsection about Major Claims.
% Please provide your estimates of human- and compute-time for each of the listed
% experiments (using the suggested hardware/software configuration above). Follows
% an example:}
All experiments below follow the same testing logic:
we craft different payloads that will be either exempted or blocked by the GFW.
We send them, from VPS in China to the VPS in US, through the GFW 
to observe whether each payload can trigger the blocking or not.
If the blocking or exemption results match with
what Algorithm 1 predicts,
it shows that the GFW is indeed using the detection rule described in Algorithm 1.
%
For reviewers' convenience, we implement Algorithm 1 as \texttt{utils/detect.py},
which reads a list of given payloads and 
writes if each payload will be exempted by any of the detection rules.

Unless explicitly specified, 
all operations described below are performed on the VPS in China.
And $\$serverIP$ corresponds to the IP address of the VPS in US.

% use paralist for more compact list format: for more details check here:
% https://texfaq.org/FAQ-complist
\begin{compactdesc}

    \item[(E1):] \textit{[test-random] [5 human-minutes + 5 compute-minutes]:
    } This experiment tests if the GFW blocks random traffic.
    It also familiarizes the reviewers with the testing tools and logic.
    \begin{asparadesc}
        \item[Preparation:] \texttt{cd artifacts}
        \item[Execution:]  
        Execute this command to generate a random probe of 200 bytes and
        check if Algorithm 1 thinks it will be blocked by the GFW or not:
        \texttt{head -c200 /dev/urandom | xxd -p -c256 | tee random.txt | ./utils/detect.py}.
        
        Execute this command to repeatedly send the probe to the same port of the US server:
        \texttt{cat random.txt | ./utils/affected-payload -host \$serverIP -p \$serverPort}.

        \item[Results:] 
        If the \texttt{affected} field in the program output is \texttt{True},
        it means that your generated probe has triggered the blocking by the GFW. 
        This result should be consistent with what \texttt{detect.py} predicts.
    \end{asparadesc}

    \item[(E2):] \textit{[test-entropy] [30 human-minutes + 30 compute-minutes]:}
    This experiment test if the GFW exempts a connection whose
    first TCP packet $\mathtt{pkt}$ satisfies:
    $\frac{\mathit{popcount}(\mathtt{pkt})}{\mathit{len}(\mathtt{pkt})} \le 3.4$ or 
    $\frac{\mathit{popcount}(\mathtt{pkt})}{\mathit{len}(\mathtt{pkt})} \ge 4.6$. 

    \begin{asparadesc}
        \item[Preparation:] \texttt{cd test-entropy}

        \item[Execution:]
        Execute this command to generate a list of payloads: \texttt{make}. 
        As shown in the output of \texttt{detect.py},
        some of the probes will be exempted by the GFW; 
        while other probes will not.

        Use this command to test if each probe is exempted by the GFW:
        \texttt{make test}.

        Use this to compare the blocking results against the results predicted by the \texttt{detect.py}:
        \texttt{make compare}.

        \item[Results:] The testing results should match with what \texttt{detect.py} predicts.
    \end{asparadesc}


    \item[(E3):] \textit{[test-printable-prefixes] [15 human-minutes + 30 compute-minutes]:}
    This experiment tests if the GFW exempts a connection whose
    first six bytes are printable characters.
    \begin{asparadesc}
        \item[Preparation:] \texttt{cd test-printable-prefixes}

        \item[Execution:]
        Execute this command to generate a list of payloads: \texttt{make}. 
        As shown in the output of \texttt{detect.py},
        some of the probes will be exempted by the GFW; 
        while other probes will not.

        Use this command to test if each probe is exempted by the GFW:
        \texttt{make test}.

        Use this to compare the blocking results against the results predicted by the \texttt{detect.py}:
        \texttt{make compare}.

        \item[Results:] The testing results should match with what \texttt{detect.py} predicts.
    \end{asparadesc}
    
    \item[(E4):] \textit{[test-printable-fraction] [15 human-minutes + 30 compute-minutes]:}
    This experiment tests if the GFW exempts a connection whose
    first TCP data packet has more than 50\% of printable characters.
    \begin{asparadesc}
        \item[Preparation:] \texttt{cd test-printable-fraction}

        \item[Execution:]
        Execute this command to generate a list of payloads: \texttt{make}. 
        As shown in the output of \texttt{detect.py},
        some of the probes will be exempted by the GFW; 
        while other probes will not.

        Use this command to test if each probe is exempted by the GFW:
        \texttt{make test}.

        Use this to compare the blocking results against the results predicted by the \texttt{detect.py}:
        \texttt{make compare}.

        \item[Results:] The testing results should match with what \texttt{detect.py} predicts.
    \end{asparadesc}
    
    \item[(E5):] \textit{[test-printable-longest-run] [15 human-minutes + 15 compute-minutes]:}
    This experiment tests if the GFW exempts a connection whose
    first TCP data packet has more than 20 bytes of contiguous printable characters.
    \begin{asparadesc}
        \item[Preparation:] \texttt{cd test-printable-longest-run}

        \item[Execution:]
        Execute this command to generate a list of payloads: \texttt{make}. 
        As shown in the output of \texttt{detect.py},
        some of the probes will be exempted by the GFW; 
        while other probes will not.

        Use this command to test if each probe is exempted by the GFW:
        \texttt{make test}.

        Use this to compare the blocking results against the results predicted by the \texttt{detect.py}:
        \texttt{make compare}.

        \item[Results:] The testing results should match with what \texttt{detect.py} predicts.
    \end{asparadesc}
    
    \item[(E6):] \textit{[test-protocol-fingerprints] [15 human-minutes + 2 compute-hours]:}
    This experiment tests if the GFW exempts traffic that matches the protocol fingerprints.
    \begin{asparadesc}
        \item[Preparation:] \texttt{cd test-protocol-fingerprints}

        \item[Execution:]
        Execute this command to generate a list of payloads: \texttt{make}. 
        As shown in the output of \texttt{detect.py},
        some of the probes start with
        a fingerprint that will be exempted by the GFW; 
        while other probes do not.

        Use this command to test if each probe is exempted by the GFW:
        \texttt{make test}.

        Use this to compare the blocking results against the results predicted by the \texttt{detect.py}:
        \texttt{make compare}.

        \item[Results:] The testing results should match with what \texttt{detect.py} predicts.
    \end{asparadesc}
    

\end{compactdesc}

% \textit{In all of the above blocks, please provide indications about the
%  expected outcome for each of the steps (given the suggested hardware/software
%  configuration above).}

%%%%%%%%%%%%%%%%%%%%%%%%%%%%%%%%%%%%%%%%%%%%%%%%%%%%%%%%%%%%%%%%%%%%%
% \subsection{Notes on Reusability}
% \label{sec:reuse}
% {\em [Optional]} \textit{This section is meant to optionally share additional
% information on how to use your artifact beyond the research presented in your
% paper. In fact, a broader objective of an artifact evaluation is to help you
% make your research reusable by others.}

% \textit{You can include in this section any sort of instruction that you believe
% would help others re-use your artifact, like, for example, scaling down/up
% certain components of your artifact, working on different kinds of input or
% data-set, customizing the behavior replacing a specific module/algorithm, etc.}

%%%%%%%%%%%%%%%%%%%%%%%%%%%%%%%%%%%%%%%%%%%%%%%%%%%%%%%%%%%%%%%%%%%%%

\subsection{Version}
%%%%%%%%%%%%%%%%%%%%
% Obligatory.
% Do not change/remove.
%%%%%%%%%%%%%%%%%%%%
Based on the LaTeX template for Artifact Evaluation V20220926. Submission,
reviewing and badging methodology followed for the evaluation of this artifact
can be found at \url{https://secartifacts.github.io/usenixsec2023/}.